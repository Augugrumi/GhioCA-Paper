\section{Know Application problems}

\subsection{Problems with Services}

We encountered different problems with some services, that are described below.

\subsubsection{Azure Image Recognition}

Azure Image Recognition is core of GhioCA. It performs image 
recognition providing a textual description of the image, and also some labels
of the content. It is a very advanced service. Unfortunately, we had some problems
with the service provider: firstly, it banned our first account; secondly it did
not accept our student subscription that would have enabled us to use it.
Finally, without any notification, Microsoft restored our first account and we
where able to use the Image Recognition service.

\subsubsection{Free OCR Space Character Recognition}

Free OCR Space is an OCR\footnote{\textit{Optical Character 
Recognition}} service that works very well, providing the text contained in a picture.
It comes with different kind of subscriptions: the free one allowed us to upload
images up to 1MB. 
We solved this limitation using different providers combination in order to use
Free OCR Space when the image is smaller than 1MB, and other services 
when it is bigger than this threshold. % TODO what are the names of the other services...?

\subsubsection{IBM Watson Image Recognition}

We also used an Image Recognition service from IBM in order to obtain more tags
from a shot performed by the user. % TODO need some rewording
Although we did not have problem with the service per-se, we found a bug in the 
Watson SDK related to the  used data format, that we where able to fix. % TODO write about the fix

\subsection{Problems with Smartphones}

During GhioCA development we found problems with particular smartphones, that
will be listed here.

\subsection{S3 first application run}

We found that with the Samsung S3 (I9300) model permissions were not granted
correctly, resulting in the impossibility to save the users' photos and thus to
perform reverse image search only at the first application run. Debugging did
not produce any evidence of bugs in our application, since the permissions were
granted during App initialization.
We address the issue to the custom ROM flashed on the smartphone, LineageOS
(Android 7.0 Nougat), that probably has some problem handling authorization on
S3 smartphones (since we tried this ROM on a Nexus 5 and we haven't noticed any
strange behavior).
To confirm our guesses we tried another camera available on the Google Play
Store, OpenCamera\footnote{For more infos:
\url{https://play.google.com/store/apps/details?id=net.sourceforge.opencamera}},
 an open-source camera that requires Android 4.0.3 (Ice Cream Sandwich) or
later.
Opening it for the first time, granting the permission to write in the external
storage and taking a photo hasn't worked even for this application, confirming
our guesses about a bug in the LineageOS S3 version.
We were not able to solve this issue without restarting the application.
