% As a general rule, do not put math, special symbols or citations
% in the abstract
\begin{abstract}
Online social networks (OSN) have revolutionized many aspects of our daily lives and have become the predominant platform where content is consumed and produced. This trend coupled with recent advances in the field of 
Artificial Intelligence (AI) have paved for many interesting added features, enriching user experience in these social platforms. 
Photo sharing and tagging is an important activity contributing to the social media data ecosystem. These data once labeled (tagged) constitute a fruitful input for the system which is exploited to better the 
services toward the user. However, these labeling process is imperfect and user subjective, hence prone to errors inherent to the process. 
In this paper, we present the design and the analysis of an Android app (namely GHio-Ca), an automatic photo tagging service relying on state-of-the-art image recognition APIs.
The application is presented to the user as a camera app used to share pictures on social networks while relying on external services to automatically retrieve tags best representing the picture theme.
Along with the system description we present a user evaluation of the system.
\end{abstract}

% keywords
\begin{IEEEkeywords}
 Online Social Networks, Social Media Sensing, Computer Vision, Android, OCR, Image Recognition.
\end{IEEEkeywords}