\section{Related work}
\label{sec:related}

Computer Vision is a branch of machine learning which is rapidly increasing, 
finding a fertile ground in many applications. One of the most obvious is 
\textit{image processing}, which involves applying changes to an image using an 
algorithm. In~\cite{Rain}, researchers developed a DNN used to remove rain 
drops from images. As in this case, many other changes can be applied to 
images: increasing size with minimum quality loss, removing objects, improving 
image resolution, and so on. However, computer vision contains also another kind 
of application: \textit{image detection}. Using NNs, algorithms are able to 
detect pedestrians in images\cite{Pedestrian} (which is fundamental for 
self-driving cars) and also detect anomalous behaviors in crowds\cite{Crowd}. 
Being able to detect shapes (e.g. a person) and irregular patterns can also 
help in terrorism detection and prevention.

On the other hand, image classification has two main problems: \textit{time} 
and \textit{cost}. This task can be performed by a machine but, in order to do 
so, the algorithm needs to be previously trained with a supervised learning 
approach. Consequently, someone has to label different images. This task is 
typically done by people who are paid to execute it. Differently from machines, 
humans are far from fast to perform image labeling; consequently, in order to 
obtain a large dataset, a lot of time is required.
As in other cases in Computer Science, \textit{parallelization} can improve the 
performance of this job: if a person needs two seconds to label an image, one 
thousand people can provide one thousand labeled images in the same amount of 
time. One good example of this principle is the database 
ImageNet\cite{ImageNet2}, ``a large-scale ontology of images built upon the 
backbone of the WordNet structure"\cite{ImageNet1}. This database can be used 
as benchmark and improvement tool for computer vision algorithms.
However, in our opinion even this last case lacks of two fundamental 
properties: \textit{usability} and \textit{zero-day learning}. Obviously, 
accessing to such database in order to retrieve information and perform pattern 
recognition is not a task which can be performed by a common user (i.e. a 
person who does not have any skill in computer science and programming) without 
a proper user interface. Furthermore, this database needs to be populated in the 
first place: this task requires paying people to do so.

Our application aims to solve these big issues, providing a usable interface 
for image recognition and exploiting smartphones to build a database from 
third-party services.
