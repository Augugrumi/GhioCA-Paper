\section{GHio-Ca}
In this section we will provide some salient features implemented by our 
application, mainly regarding the architectural and implementation choices that 
we made developing the code.

\subsection{Architecture}
GHio-Ca is principally composed of two layers: (i) the layer that have to manage
network connectivity, uploading photos to the server and making requests to 
different services in order to make the image recognition or the translation of 
certain pieces of text, and (ii) the camera layer, that manages the photo 
capturing process and saving. 

These two layers are maintained as loose coupled as possible in order to make 
easy to change services used without making important changes to the overall 
application or changing the process of photo taking, without modify the 
networking layer.

%TODO
%	-add a UML of the application architecture at really high level
%		   NETWORK
%		
%		|-----------|           |---------------|
%		| LISTENERS |------>    |     PHOTO     |
%		| ASYNCTASK |<------    |               |
%		|-----------|           |---------------|

\subsection{Application description}
GHio-Ca (Giving Hashtags In Order To Classify Automatically) is an Android 
application that gives to a user the possibility to make image recognition on 
her pictures. This application embed some camera features in order to make 
possible to take a photo and recognize it directly, but it also allows to pick 
a picture from the saved ones and launch the classifying process on it.
GHio-Ca is specifically designed to recognize objects and text: the only 
limitation is that the user has to choose it before starting the 
recognition process from the hamburger menu placed in the main activity. 
The overall application is mainly composed by four activities:
\begin{itemize}
  \item splash activity: this view is needed in order to gain from the user
    the necessary permissions (network access, camera access, access to 
    the storage);
  \item camera preview activity: this one displays the user a camera
    interface, with the possibility to take photos, turn off the flash, switch 
    to frontal camera (if present) and to access to the gallery in order to 
    pick a file instead of taking a picture. In the top left corner is present 
    the hamburger menu: it allows user to choose the size of picture taken, 
    it reminds the user to turn on the Wi-Fi sensor on every access and it 
    allows to switch from the image recognition functionality to the 
    character optical recognition feature and vice-versa;
  \item results activities: these are two different activity depending on 
    the type of image recognition chosen. If "reverse image search" is 
    selected the result activity will be presented. Without errors 
    occurs, a textual description of the picture, with a bunch of hashtags 
    correlated to it. If "character recognition" is chosen, the result activity 
    will present the text recognized. % TODO fixme!
\end{itemize}
