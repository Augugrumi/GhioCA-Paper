\section{Future work}
\label{sec:future}

Working on this project led us to new features that we can implement in the future.
One possible extension of our work is to create our own dataset of images and use it in order to classify images. This task could be easily accomplished using services such Clarifai\footnote{\url{https://www.clarifai.com/}}. This service would allow us to create our own set of pictures and tags, and run images recognition algorithm using these images as training set. In order to create the initial set of pictures, it is possible to use the application "\textit{as it is}", distributing it to final users. In this way, exploiting the services, we can be able to create a private dataset. In addition, this feature would enable us to become independent from external services.
Moreover, another possibility is to specialize the recognition of a specific kind of object. Following this idea, we thought as one possible application to recognize monuments and other historical places around different cities. In order to optimize the recognition process of historical monuments, it is certainly useful to take advantage of the user position. Consequently, using geographical position (i.e. GPS) we can resize training data to a certain amount of photo, improving the performance of our classifier avoiding misclassification.

Finally, this could also be used by big firms from Silicon Valley to speed up their API, since they would have access to a distributed network used by millions of users who share daily contents on social networks. All they should do is provide free access to algorithms.