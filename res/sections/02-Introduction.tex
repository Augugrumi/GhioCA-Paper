\section{Introduction}

%TODO
%	- insert some statistics of photo shared on media

Nowadays, almost everyone owns at least one smartphone. That kind of devices are
becoming more and more powerful, with better hardware and performace. One of the
most interesting capability of smartphones is to take really realistic photo, and
users truly appreciate it. As a matter of fact that on the Google's Play Store 
there are plenty of application dedicated to photography.

A lot of people take photo in order to share it on social networks, such as 
Facebook, Instagram, Twitter and so on, and often those are photos are 
accompanied by a description and, generally, some hashtags that label them. The 
latters could be personal words that a person associate to a certain photo but it
also could be used in order to describe the content of the photo.

All of this pieces of information could be really useful to train algorithms and
to create datasets used in order to recognize images: in fact, these data, freely 
available,are posted by a person that manually labeled and described a specific 
photo.

Usually, image recognition services process an image, giving back a result that 
consists of a set of tags correlated to that picture. Results could be more or 
less precise depending on the photo quality, subject and algorithm used. The 
result given back is almost always is not perfect: as a metter of fact that, at 
the state of art, image recognition process could not as precise as could be a 
human to describe a picture.

One of the opportunities to profit by this context is to embed in an application
the possibility of taking photos that will be processed by some image recognition 
service. The result of this process will be displayed to the user, that could 
choose only the subset of tags that are truly correlated to her picture: in that
way users can automatically associate some tags to their photos and share them on
social networks. On the other hand, with a human check to the result, you could 
evaluate how much a specific service work well or train your own dataset or 
algorithm. 

Following this idea, we have designed and developed GHio-Ca(Giving Hashtags In 
Order to Classify Automatically), a simple Android application that allows people
to take photo or choose picture that will be automatically prosessed by some 
image recognition services. In particular we have utilized these services:
\begin{itemize}
	\item Azure;
	\item Watson;
	\item Google reverse image search;
	\item Immaga;
	\item OCR Space.
\end{itemize}

While the quality of service and the usability are major issues when developing 
an application, we have focused principally on technical aspect, such as will be 
able to correctly take photo and save them, make request to image recognition 
services and parse result in order to display them. We have not focused our 
affords on robustness of the application, so if the request could not be 
performed (for example for a network failure or a service fault) the application 
only notify to the user that some error occurred, however GHio-Ca allows user to 
retry to retry the image recognition process on the picture.






